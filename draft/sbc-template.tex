\documentclass[12pt]{article}

\usepackage{sbc-template}

\usepackage{graphicx,url}

\usepackage[T1]{fontenc}
\usepackage[utf8]{inputenc}
\usepackage{lmodern}
\usepackage[brazil]{babel}

\usepackage{verbatim}
     
\sloppy

\title{Metodologia de Redução de Custos no Planejamento de Redes de Acesso PON}

\author{Kelvin Batista da Cunha\\
Marcus Felipe Raele Rios\\
Otávio Lucas Alves da Silva
}

\address{Centro de Informática - UFPE}

\begin{document} 

\maketitle
\begin{resumo} 
Este rascunho tem como objetivo  descrever brevemente uma metodologia de implementação de redes de acesso PON.
Serão utilizadas heurísticas de otimização buscando minimizar \textit{Capital Expenditures} (CAPEX) e a partir do modelo obtido, estimar 
o \textit{Operational Expenditures} (OPEX) a partir de um intervalo de confiança.
\end{resumo}


\section{Descrição do Problema}

A ausência de metodologias de planejamento de redes de acesso PON em regiões  urbanas pode acarretar em redes baseadas apenas na experiência prévia do planejador. Embora  na maior parte das vezes distribuição atenda de forma satisfatória a demanda por largura de banda, os CAPEX e OPEX associados geralmente não são os melhores que uma operadora poderia assumir. Durante o texto, iremos modelar a minização do CAPEX como um problema de otimização, definir uma heurística para encontrar uma solução subótima e a partir desta, definir os custos reais da implantação da rede, incluindo fatores como trajetórias dos cabos nas vias, custo de mão de obra, OLTs, ONUs etc.

A metodologia apresentada considerará o modelo \textit{Fiber to The Home} (FTTH), i.e., cada residência possuirá sua própria ONU. Desta forma, não precisaremos considerar  tecnologias híbridas, como a \textit{Hybrid Fiber Coaxial} (HFC), simplificando a modelagem.

A heurística e a modelagem aqui apresentadas, são fortemente baseada nas do artigo ???. Nele, é descrito o \textit{Recursive Association and Relocation Algorithm} (RARA), algoritmo que retorna uma conjunto de splitters e suas respectivas associações com ONUs. Todas as restrições do problema de otimização devem ser respeitadas. Por fim, iremos apresentar um algoritmo, que, dada a localização dos splitters e suas associações com as ONUs, retorne a trajetória nas vias por onde passarão as fibras. Particularmente, este algoritmo considera fatores mais realistas que a modelagem do problema de otimização. Veremos quais mais adiante.

\subsection{Modelagem do Problema}

As principais restrições do problema de otimização podem ser divididas em três categorias:
\begin{itemize}
\item Restrições que se referem ao funcionamento de uma rede de acesso PON: ONUs precisam se conectar a splitters, e splitters precisam se conectar a splitters de mais alta ordem ou a OLTs; um splitter não poderá disponibilizar mais canais que sua capacidade etc.
\item Restrições
\end{itemize}

\bibliographystyle{sbc}
\bibliography{sbc-template}

\end{document}
